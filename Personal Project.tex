\documentclass[11pt]{article}
\usepackage{fullpage}
\usepackage{graphicx}
\usepackage{color}
\title{\textbf{Personal Project: Maze with Turtle}}
\author{Ersin Kunada}
\date{}

\begin{document}
\maketitle

\section*{Introduction}
In my project I am going to create my own version of the game that is commonly known as \textit{Maze}. The game will be implemented and played in the module Turtle of Python.

\section*{Gameplay}
The aim of the game consists in going from the start to the end of the maze without losing the way.
The program should create a random maze in any match and the player should try every time to reach the ending of the maze by giving inputs with the keyboard. To play the game, the player can use 4 commands which correspond to the arrows in the keyboard. It is not possible to pass from a point that the player has already passed. If the player reaches a blind corner, the game restarts with a new random maze.

\section*{Maze}
The maze I am going to use is a square grid (10 cells by 10 cells). The walls of the maze are coloured in black, whereas the path is white. The starting cell is at the bottom left (x=0 and y=0) and the ending cell is at the top right (x=10 and y=10); both cells are coloured in red. When the player passes for a cell, this change colour and becomes blue.

\section*{Features}
In this paragraph I am going to summarise all the features of my game:
\begin{itemize}
\item Random maze 
\item 4 commands to control the turtle
\item No possibility to turn back
\item Restart the game when a blind corner is reached
\item 10 x 10 cells for the maze
\item Fixed starting and ending point
\item End the game when the end is reached
\end{itemize}


\end{document}


